%%%%%%%%%%%%%%%%%%%%%%%%%%%%%%%%%%%%%%%%%%%%%%%%%%%%%%%%%%%%%%
% Sweave script for wgeel recruitment analysis
% Author Cedric Briand / Estibaliz Diaz
% you can run this script with command sweave() or appropriate tools
% such as Eclipse or Rstudio
% this script is intended to run with sweave not knitr
% It produces a nice pdf which has to be turned into a less nice word document
% The problem is that the encoding of my files is Cp1252
% The only way out is a full shift to utf8, in init pass locale for R to Sys.setlocale("LC_ALL", "English_United States.932")
% In the preamble replace \usepackage[latin1]{inputenc} by \usepackage[utf8]{inputenc}
% Note the following does not work :
% iconv -f CP1252 -t UTF-8 recruitment_analysis.tex > recruitment_analysis_utf8.tex
% finally the command to run is : 
% cd C:\workspace\gitwgeel\R\recruitment
% pandoc -s -o recruitment.docx --from=latex --to=docx  recruitment_analysis.tex
% All this for producing a ugly word report :-( 
%%%%%%%%%%%%%%%%%%%%%%%%%%%%%%%%%%%%%%%%%%%%%%%%%%%%%%%%%%%%%%

%%%%%%%%%%%%%%%%%%%%%%%%%%%%%%%%%%%%%%%%%%%%%%%%%%%%%%%%%%%%%%%%%%%%%%%%%%%%%%%%%%%%%
%************************************************************************************
%%%%%%%%%%%%%%%%%%%%%%%%%%%%%%%%%%%%%%%%%%%%%%%%%%%%%%%%%%%%%%%%%%%%%%%%%%%%%%%%%%%%%%

%%%%%%%%%%%%%%%%%%%%%%%%%%%%%%%%%%%%%%%%%%%%%%%%%%%%%%%%%%%%%%%%%
% this is the preamble of the latex file
% it is used to load the packages necessary to the analysis
%%%%%%%%%%%%%%%%%%%%%%%%%%%%%%%%%%%%%%%%%%%%%%%%%%%%%%%%%%%%%%%%%
ruitme\documentclass[pdftex,11pt,a4paper]{report}
\usepackage[pdftex]{graphicx}
\usepackage[utf8]{inputenc} %encodage du fichier source
%\usepackage[T1]{fontenc}  %gestion des accents (pour les pdf) 
\usepackage[english]{babel}
\usepackage[swedish]{babel}
\usepackage{Sweave}
\usepackage{float}
\usepackage[left=3cm, right=3cm, top=2cm, bottom=2cm]{geometry}
\usepackage{pdflscape}% to rotate the table
\geometry{dvips,a4paper,hmargin=2.5cm,vmargin=2.5cm}
\setcounter{secnumdepth}{10}
\usepackage{hyperref} %managing hyperlinks
\hypersetup{
     backref=true,    %permet d'ajouter des liens dans...
     pagebackref=true,%...les bibliographies
     hyperindex=true, %ajoute des liens dans les index.
     colorlinks=true, %colorise les liens
     breaklinks=true, %permet le retour <U+FFFD> la ligne dans les liens trop longs
     urlcolor= blue,  %couleur des hyperliens
     linkcolor= blue, %couleur des liens internes
     bookmarks=true,  %cr<U+FFFD><U+FFFD> des signets pour Acrobat
     bookmarksopen=true,            %si les signets Acrobat sont cr<U+FFFD><U+FFFD>s,
                                    %les afficher compl<U+FFFD>tement.
     pdftitle={WGEEL recruitment analysis output}, %informations apparaissant dans
     pdfauthor={C<U+FFFD>dric Briand},     %dans les informations du document
     pdfsubject={Glass eel recruitment}          %sous Acrobat
     pdfkeywords={Glass eel, WGEEL, recruitment, time-series, ICES}
}
\usepackage{wrapfig}
\usepackage{tabularx}
\usepackage{longtable}
\usepackage[table,usenames,dvipsnames]{xcolor}
\newcolumntype{R}{>{\raggedleft\arraybackslash}X}
\newcolumntype{L}{>{\raggedright\arraybackslash}X}
% this is where I'm storing files locally
\graphicspath{{C:/workspace/gitwgeel/R/recruitment/2020/image//}} 
% TO UPDATE TO THE NEW YEAR
% DO REPLACE FOR INSTANCE F:/workspace/wgeeldata/sweave/2016/table by F:/workspace/wgeeldata/sweave/2019/table
%%%%%%%%%%%%%%%%%%%%%%%%%% END OF THE PREAMBLE  %%%%%%%%%%%%%%%%%%%%%%%%%%%%%%%%%%%%%%%%%%%%%%%%%%%%%%%%%%%%%%
\begin{document}
\input{recruitment_analysis-concordance}
\bibliographystyle{plain}
\title{ Analysis of recruitment trend}
\author{EIFAAC/ICES/GFCM Working Group on Eel}
\date\today
\maketitle
\newpage
\tableofcontents
\subsection*{}
% use this if the chair wants a different chapter heading
%\addtocounter{chapter}{3} % this will set initial chapter counter to  4
%======================================================================
\addtocounter{chapter}{1}
\chapter{Report on developments on the state of the European eel (Anguilla
anguilla) stock, the fisheries on it and other anthropogenic impacts.}



%%%%%%%%%%%%%%%%%%%%%%%%%%%%%%%%%%%%%%%%%%%%%%%%%%%%%%%%%%%%%%%%%%%%%%%%%%%%%%%%%%%%%%
%*******Here put eval=FALSE if you plan to use Rdata *********************************
%%%%%%%%%%%%%%%%%%%%%%%%%%%%%%%%%%%%%%%%%%%%%%%%%%%%%%%%%%%%%%%%%%%%%%%%%%%%%%%%%%%%%%

%%%%%%%%%%%%%%%%%%%%%%%%%%%%%%%%%%%%%%%%%%%%%%%%%%%%%%%%%%%%%%%%%%%%%%%%%%%%%%%%%%%%%%
%************************************************************************************
%%%%%%%%%%%%%%%%%%%%%%%%%%%%%%%%%%%%%%%%%%%%%%%%%%%%%%%%%%%%%%%%%%%%%%%%%%%%%%%%%%%%%%
%%%%%%%%%%%%%%%%%%%%%%%%%%%%%%%%%%%%%%%%%%%%%%%%%%%%%%%%%%%%%%%%%%%%%%%%%%%%%%%%%%%%%%
% 				SERIES SELECTION AND STATISTICS ABOUT DATA ENTRY 
% the statistics about the series are usefull while filling in new data, to check
% which series have been already entered and which still need to be
%%%%%%%%%%%%%%%%%%%%%%%%%%%%%%%%%%%%%%%%%%%%%%%%%%%%%%%%%%%%%%%%%%%%%%%%%%%%%%%%%%%%%%

%%%%%%%%%%%%%%%%%%%%%%%%%%%%%%%%%%%%%%%%%%%%%%%%%%%%%%%%%%%%%%%%%%%%%%%%%%%%%%%%%%%%%%
%				THIS CHUNK CREATES THE TABLES	
%%%%%%%%%%%%%%%%%%%%%%%%%%%%%%%%%%%%%%%%%%%%%%%%%%%%%%%%%%%%%%%%%%%%%%%%%%%%%%%%%%%%%%

%%%%%%%%%%%%%%%%%%%%%%%%%%%%%%%%%%%%%%%%%%%%%%%%%%%%%%%%%%%%%%%%%%%%%%%%%%%%%%%%%%%%%%
%************************************************************************************
%%%%%%%%%%%%%%%%%%%%%%%%%%%%%%%%%%%%%%%%%%%%%%%%%%%%%%%%%%%%%%%%%%%%%%%%%%%%%%%%%%%%%%
%%%%%%%%%%%%%%%%%%%%%%%%%%%%%%%%%%%%%%%%%%%%%%%%%%%%%%%%%%%%%%%%%%%%%%%%%%%%%%%%%%%%%%
%************************************************************************************
%%%%%%%%%%%%%%%%%%%%%%%%%%%%%%%%%%%%%%%%%%%%%%%%%%%%%%%%%%%%%%%%%%%%%%%%%%%%%%%%%%%%%%
%%%%%%%%%%%%%%%%%%%%%%%%%%%%%%%%%%%%%%%%%%%%%%%%%%%%%%%%%%%%%%%%%%%%%%%%%%%%%%%%%%%%%%
%************************************************************************************
%%%%%%%%%%%%%%%%%%%%%%%%%%%%%%%%%%%%%%%%%%%%%%%%%%%%%%%%%%%%%%%%%%%%%%%%%%%%%%%%%%%%%%
%%%%%%%%%%%%%%%%%%%%%%%%%%%%%%%%%%%%%%%%%%%%%%%%%%%%%%%%%%%%%%%%%%%%%%%%%%%%%%%%%%%%%%
%************************************************************************************
%%%%%%%%%%%%%%%%%%%%%%%%%%%%%%%%%%%%%%%%%%%%%%%%%%%%%%%%%%%%%%%%%%%%%%%%%%%%%%%%%%%%%%



\section{Data sources}
\section{Recruitment}

In this section, the latest trends in glass and yellow eel recruitment are
addressed. The time-series data are derived from fishery-dependent sources
(i.e. catch records) and also from fishery-independent surveys across much of
the geographic range of European eel.  The
stages are categorized as :
\begin{itemize}
\item glass eel (G), continental age 0 years, 
\item a mixture of glass eel and young
yellow eel dominated by recruits from the same year (GY), and
\item older yellow eel (Y) recruiting to continental habitats. The yellow
eel series might consist of yellow eel of several ages.
This is certainly the case for all series from the Baltic (mean age up to 6),
 some Irish sites, and sites located far upstream.
\end{itemize}
The glass eel recruitment time-series have been grouped into two geographical
areas: 'continental North Sea' (NS) and 'Elsewhere Europe' (EE) (Figure
\ref{figure1}). The glass eel recruitment time-series have been grouped into two geographical
areas: 'continental North Sea' (NS) and 'Elsewhere Europe' (EE) (Figure 3.4.1).
Previous analyses by the working group (ICES, 2010b, p19, Bornarel et al. (2017) 
have shown a different trend between the two sets. This is mostly due to a more pronounced
decline of the North Sea series compared to the Elsewhere Europe area during the 1980s.

The WGEEL has collated information on recruitment from
$95$ time-series. Some time-series date back to the
beginning of 20th century (yellow eel, Gota Alv, Sweden) or 1920 (glass eel,
Loire, France). Among those series $68$ have been
selected for further analysis in the wgeel indices; see details on data
selection and processing below. Depending on the standardization period, the number of series  used can be lower and is given for
each analysis. 

%==================================================
\begin{figure}[htbp]
\centering
\includegraphics[width=0.7\textwidth]{figure1.png}
\caption{Map of recruitment sampling stations, color according to stage (grey =
G and GY) yellow = Y.}
\label{figure1}
\end{figure}
%==========================================

\subsection{Details on data selection and processing}

Out of $95$, 
$68$ series were used in the analysis. Three
rules have been used for this selection procedure.
\begin{enumerate}
\item First, if there are two or more series from the same location, i.e. they are not independent,only one series is kept. For instance, the longer series has been kept for the Severn (Severn EA) while the other series (Severn HMRC) has been dropped from the list, as it was considered a duplicate being based on the same fishery.Noting that the 'Severn' here actually represents all the glass eel fisheries for England and Wales but the naming convention has been used for many
years so is retained for consistency.
\textcolor{Plum}{TODO work on Minho sp and Minho
pt. and Gironde....}
\item The second rule is to exclude a series from the analysis when it is less than ten years long. The series are, however, still updated in the database until they are long enough to be included. \textcolor{Plum}{But clarify now what we are going to do
when there are missing years....} 
\item Finally, it was decided to discard recruitment series that were obviously biased by restocking (e.g. Farpener Bach in Germany).
\end{enumerate}


Among the time-series based on trap indices, some have reported
preliminary data for 2020 as their trapping season had not
finished. \textbf{As usual, the indices given for 2020 must be considered
as provisional especially those for the yellow eel.}



 %====================================
	% latex table generated in R 4.1.0 by xtable 1.8-4 package
% Wed Sep 15 09:22:35 2021
\begin{table}[htbp]
\centering
\caption{Short description of the sampling sites for European eel recruitment data. Area: NS = North Sea, EE = Elsewhere Europe. Min and max indicate the first year and last year in the records, and the values are given in the n+ and n- columns, indicate the number of years with values and the number of years when there are missing data within the series. Life stage: GY = glass eel and yellow eel, G = glass eel, Y = yellow eel. Unit for the data collected is given (nr = number; index = calculated value following a specified protocol, nr/m2 = number per square metre, nr/h = number per hour, kg/boat/d = kg per boat per day). Habitat: C = coastal water (according to the EU Water Framework Directive, WFD), F = freshwater, MO = marine water (open sea), T = transitional water with lower salinity (according to WFD). Kept = 1 means that the dataseries is used in recruitment analyses.} 
\label{statseriesG}
\begin{tabular}{p{1cm}p{1cm}p{1cm}p{1cm}p{0.8cm}p{0.8cm}p{1cm}p{2cm}p{2cm}p{1cm}p{1cm}}
  \hline
code & area & min & max & n+ & n- & life stage & sampling type & unit & habitat & kept \\ 
  \hline
RingG & NS & 1981 & 2021 & 41 & 0 & G & sci. surv. & index & C & 1 \\ 
  YFS1G & NS & 1975 & 1989 & 15 & 0 & G & sci. surv. & index & MO & 1 \\ 
  YFS2G & NS & 1991 & 2021 & 31 & 0 & G & sci. surv. & index & MO & 1 \\ 
  EmsG & NS & 1946 & 2001 & 56 & 0 & G & com. catch & kg & T & 1 \\ 
  EmsHG & NS & 2011 & 2020 & 10 & 0 & G & trap & nr & T & 0 \\ 
  WaSG & NS & 2011 & 2020 & 10 & 0 & G & sci. surv. & nr & T & 0 \\ 
  KlitG & NS & 2008 & 2021 & 14 & 0 & G & sci. surv. & nr/m2 & F & 1 \\ 
  NorsG & NS & 2008 & 2021 & 14 & 0 & G & sci. surv. & nr/m2 & F & 1 \\ 
  SleG & NS & 2008 & 2021 & 14 & 0 & G & sci. surv. & nr/m2 & F & 1 \\ 
  VidaG & NS & 1971 & 1990 & 20 & 0 & G & com. catch & kg & T & 1 \\ 
  KatwG & NS & 1977 & 2021 & 45 & 5 & G & sci. surv. & index & T & 1 \\ 
  LauwG & NS & 1976 & 2021 & 46 & 4 & G & sci. surv. & nr/h & T & 1 \\ 
  RhDOG & NS & 1938 & 2021 & 84 & 1 & G & sci. surv. & index & T & 1 \\ 
  RhIjG & NS & 1969 & 2021 & 53 & 5 & G & sci. surv. & index & T & 1 \\ 
  StelG & NS & 1971 & 2021 & 51 & 0 & G & sci. surv. & index & T & 1 \\ 
  VeAmG & NS & 2017 & 2021 & 5 & 0 & G & trap & kg & T & 0 \\ 
  YserG & NS & 1964 & 2021 & 58 & 1 & G & sci. surv. & kg & T & 1 \\ 
  BurrG & EE & 1987 & 2021 & 35 & 18 & G & trap & kg & F & 1 \\ 
  MaigG & EE & 1994 & 2021 & 28 & 4 & G & trap & kg & F & 1 \\ 
  BeeG & EE & 2006 & 2020 & 15 & 0 & G & trap & nr & F & 1 \\ 
  BroG & EE & 2011 & 2021 & 11 & 0 & G & trap & nr & F & 1 \\ 
  FlaG & EE & 2007 & 2020 & 14 & 0 & G & trap & nr & F & 1 \\ 
  SeEAG & EE & 1972 & 2021 & 50 & 2 & G & com. catch & t & T & 1 \\ 
  SeHMG & EE & 1979 & 2021 & 43 & 4 & G & com. catch & t & T & 3 \\ 
  ShiFG & EE & 2011 & 2021 & 11 & 0 & G & trap & nr & F & 0 \\ 
  ShiMG & EE & 2011 & 2021 & 11 & 0 & G & trap & nr & T & 0 \\ 
  AdCPG & EE & 1928 & 2008 & 81 & 40 & G & com. cpue & kg/boat/d & T & 1 \\ 
  AdTCG & EE & 1986 & 2008 & 23 & 0 & G & com. catch & t & T & 1 \\ 
  GiCPG & EE & 1961 & 2008 & 48 & 1 & G & com. cpue & kg/boat/d & T & 1 \\ 
  GiScG & EE & 1992 & 2021 & 30 & 0 & G & sci. surv. & index & T & 1 \\ 
  GiTCG & EE & 1923 & 2008 & 86 & 28 & G & com. catch & t & T & 1 \\ 
  LoiG & EE & 1924 & 2008 & 85 & 6 & G & com. catch & kg & T & 1 \\ 
  SevNG & EE & 1962 & 2008 & 47 & 25 & G & com. cpue & kg/boat/d & T & 1 \\ 
  VacG & EE & 2004 & 2021 & 18 & 0 & G & trap & nr & T & 1 \\ 
  VilG & EE & 1971 & 2015 & 45 & 3 & G & trap & t & T & 1 \\ 
  AlbuG & EE & 1949 & 2021 & 73 & 5 & G & com. catch & kg & F & 1 \\ 
  AlCPG & EE & 1982 & 2021 & 40 & 5 & G & com. cpue & kg/boat/d & F & 1 \\ 
  EbroG & EE & 1966 & 2021 & 56 & 3 & G & com. catch & kg & T & 1 \\ 
  GuadG & EE & 1998 & 2007 & 10 & 0 & G & sci. surv. & index & T & 1 \\ 
  MiSpG & EE & 1975 & 2021 & 47 & 0 & G & com. catch & kg & T & 1 \\ 
  NaloG & EE & 1953 & 2021 & 69 & 0 & G & com. catch & kg & T & 1 \\ 
  OriaG & EE & 2006 & 2021 & 16 & 0 & G & sci. surv. & nr/m3 & T & 1 \\ 
  MiPoG & EE & 1974 & 2021 & 48 & 0 & G & com. catch & kg & T & 1 \\ 
  MiScG & EE & 2018 & 2021 & 4 & 0 & G & sci. surv. & nr/h & T & 0 \\ 
  MondG & EE & 1989 & 2021 & 33 & 28 & G & sci. surv. & nr/h & T & 0 \\ 
  TibeG & EE & 1975 & 2006 & 32 & 0 & G & com. catch & t & T & 1 \\ 
   \hline
\end{tabular}
\end{table}

 %====================================

 %====================================
	% latex table generated in R 4.1.0 by xtable 1.8-4 package
% Fri Oct 01 17:06:19 2021
\begin{table}[htbp]
\centering
\caption{Short description of the recruitment sites (continued : stage glass eel and yellow eel)} 
\label{statseriesGY}
\begin{tabular}{p{1cm}p{1cm}p{1cm}p{1cm}p{0.8cm}p{0.8cm}p{1cm}p{2cm}p{2cm}p{1cm}p{1cm}}
  \hline
code & area & min & max & n+ & n- & life stage & sampling type & unit & habitat & kept \\ 
  \hline
ImsaGY & NS & 1975 & 2021 & 47 & 0 & GY & trap & nr & F & 1 \\ 
  ViskGY & NS & 1972 & 2020 & 49 & 0 & GY & trap & kg & F & 1 \\ 
  BrokGY & NS & 2011 & 2021 & 11 & 0 & GY & trap & nr & T & 1 \\ 
  EmsBGY & NS & 2011 & 2020 & 10 & 0 & GY & trap & nr & F & 0 \\ 
  FarpGY & NS & 2007 & 2020 & 14 & 0 & GY & trap & nr & F & 3 \\ 
  HHKGY & NS & 2010 & 2021 & 12 & 0 & GY & trap & nr & T & 0 \\ 
  HoSGY & NS & 2010 & 2010 & 1 & 0 & GY & trap & nr & T & 0 \\ 
  LangGY & NS & 2011 & 2021 & 11 & 0 & GY & trap & nr & T & 0 \\ 
  VerlGY & NS & 2010 & 2021 & 12 & 0 & GY & trap & nr & T & 1 \\ 
  WiFG & NS & 2006 & 2020 & 15 & 0 & GY & trap & nr & T & 1 \\ 
  WisWGY & NS & 2004 & 2020 & 17 & 0 & GY & trap & nr & F & 1 \\ 
  HellGY & NS & 2010 & 2020 & 11 & 0 & GY & sci. surv. & nr & T & 1 \\ 
  ErneGY & EE & 1959 & 2021 & 63 & 2 & GY & trap & kg & F & 1 \\ 
  FealGY & EE & 1985 & 2021 & 37 & 14 & GY & trap & kg & F & 1 \\ 
  InagGY & EE & 1996 & 2021 & 26 & 4 & GY & trap & kg & F & 1 \\ 
  LiffGY & EE & 2011 & 2021 & 11 & 0 & GY & trap & kg & F & 1 \\ 
  ShaAGY & EE & 1977 & 2021 & 45 & 0 & GY & trap & kg & F & 1 \\ 
  BannGY & EE & 1933 & 2021 & 89 & 0 & GY & trap & kg & F & 1 \\ 
  BeeGY & NS & 2011 & 2020 & 10 & 0 & GY & trap & nr & F & 1 \\ 
  BroGY & NS & 2011 & 2021 & 11 & 0 & GY & trap & nr & F & 3 \\ 
  FlaGY & NS & 2007 & 2020 & 14 & 0 & GY & trap & nr & F & 3 \\ 
  GreyGY & EE & 2009 & 2020 & 12 & 0 & GY & trap & nr & F & 1 \\ 
  NmiGY & NS & 2009 & 2021 & 13 & 0 & GY & trap & nr & F & 1 \\ 
  OatGY & EE & 2011 & 2021 & 11 & 0 & GY & trap & nr & F & 0 \\ 
  StraGY & EE & 2011 & 2021 & 11 & 0 & GY & trap & nr & F & 1 \\ 
  BresGY & EE & 1994 & 2021 & 28 & 0 & GY & trap & nr & F & 1 \\ 
  SousGY & EE & 2013 & 2021 & 9 & 0 & GY & trap & nr & F & 0 \\ 
   \hline
\end{tabular}
\end{table}

 %====================================F

 %====================================
	% latex table generated in R 4.1.0 by xtable 1.8-4 package
% Fri Oct 01 07:00:09 2021
\begin{table}[htbp]
\centering
\caption{Short description of the recruitment sites (continued : yellow eel series) } 
\label{statseriesY}
\begin{tabular}{p{1cm}p{1cm}p{1cm}p{1cm}p{0.8cm}p{0.8cm}p{1cm}p{2cm}p{2cm}p{1cm}p{1cm}}
  \hline
code & area & min & max & n+ & n- & life stage & sampling type & unit & habitat & kept \\ 
  \hline
DalaY & NS & 1951 & 2020 & 70 & 3 & Y & trap & kg & F & 1 \\ 
  GotaY & NS & 1900 & 2020 & 121 & 12 & Y & trap & kg & F & 1 \\ 
  KavlY & NS & 1992 & 2020 & 29 & 0 & Y & trap & kg & F & 1 \\ 
  LagaY & NS & 1925 & 2020 & 96 & 0 & Y & trap & kg & F & 1 \\ 
  MorrY & NS & 1960 & 2019 & 60 & 0 & Y & trap & kg & F & 1 \\ 
  MotaY & NS & 1942 & 2020 & 79 & 0 & Y & trap & kg & F & 1 \\ 
  RonnY & NS & 1946 & 2019 & 74 & 9 & Y & trap & kg & F & 1 \\ 
  DoElY & NS & 2003 & 2020 & 18 & 0 & Y & trap & nr & F & 1 \\ 
  WaSEY & NS & 2011 & 2020 & 10 & 0 & Y & sci. surv. & nr & T & 0 \\ 
  GudeY & NS & 1980 & 2020 & 41 & 0 & Y & trap & kg & F & 1 \\ 
  HartY & NS & 1967 & 2020 & 54 & 1 & Y & trap & kg & F & 1 \\ 
  MeusY & NS & 1992 & 2020 & 29 & 3 & Y & trap & nr & F & 4 \\ 
  VeAmY & NS & 2017 & 2021 & 5 & 0 & Y & trap & nr & T & 0 \\ 
  ShaPY & EE & 1985 & 2021 & 37 & 0 & Y & trap & kg & F & 1 \\ 
  BeeY & NS & 2011 & 2020 & 10 & 0 & Y & trap & nr & F & 1 \\ 
  BroY & NS & 2011 & 2021 & 11 & 0 & Y & trap & nr & F & 1 \\ 
  FlaY & NS & 2012 & 2020 & 9 & 0 & Y & trap & nr & F & 1 \\ 
  GirnY & NS & 2008 & 2021 & 14 & 0 & Y & trap & nr & F & 1 \\ 
  MertY & NS & 2011 & 2021 & 11 & 0 & Y & trap & nr & F & 1 \\ 
  MillY & NS & 2011 & 2021 & 11 & 0 & Y & trap & nr & F & 1 \\ 
  MolY & NS & 2005 & 2021 & 17 & 0 & Y & trap & nr & F & 1 \\ 
  RodY & NS & 2005 & 2020 & 16 & 0 & Y & trap & nr & F & 1 \\ 
  FreY & EE & 1997 & 2020 & 24 & 0 & Y & trap & nr & F & 1 \\ 
  MiSpY & EE & 2019 & 2020 & 2 & 0 & Y & trap & kg & T & 0 \\ 
   \hline
\end{tabular}
\end{table}

 %====================================
 
 
  %====================================
	% latex table generated in R 4.1.0 by xtable 1.8-4 package
% Fri Oct 01 09:46:46 2021
\begin{table}[htbp]
\centering
\caption{Data in 2021 and 2020having problems causing the data in the specific year to be excluded from the analysis. Codes for stages are G = glass eel, GY = glass eel + yellow eel, Y = yellow eel, Division = FAO marine division. Kept: 0 = missing, 1 = good quality,
						2 = wgeel has modified the data, 3 = not used due to poor quality, 4 =	data is used, but there are warnings on its quality} 
\label{table_series_prob}
\begin{tabular}{p{1.5cm}p{1.5cm}p{1.5cm}p{1cm}p{1cm}p{1cm}p{8cm}}
  \hline
\scshape{Name} & \scshape{Stage} & \scshape{Country} & \scshape{Division} & \scshape{Year} & \scshape{Kept} & \scshape{Comment} \\ 
  \hline
BeeG & G & GB & 27.4.c & 2020 &   4 & Provisional data as of June 2020 - comment updated 2021 and confirmed as a final count for 2020. Das\_value updated from 7446 to 8303. Monitoring impacted by COVID19. \\ 
  BroG & G & GB & 27.4.c & 2020 &   4 & Comment updated from "Provisional data as of June 2020" to "Final count for 2020". Monitoring impacted by COVID19. \\ 
  FlaG & G & GB & 27.4.c & 2020 &   4 & Value updated 2021 from 93 to 1136. Underestimate due to impact of Covid restrictions.  \\ 
  NaloG & G & ES & 27.8.c & 2020 &   4 & In March (allowed from 20 to 27) only a few fishermen were active because of the reduced price of glass eel due to the COVID-19. \\ 
  SeEAG & G & GB & 27.7.f & 2020 &   4 & . \\ 
  SeHMG & G & GB & 27.7.f & 2020 &   4 & Note that UK trade of glass eel has been impacted by COVID19- elver station closure within season will have impacted upon effort \\ 
  ShiFG & G & GB & 27.6.a & 2020 &   0 & Covid-19 prevented collection \\ 
  VacG & G & FR & 37.1.2 & 2020 &   4 & due to COVID19, the glass eel monitoring was stop since mid-march then one month of monitoring was not made at the end of the migration period \\ 
  VeAmG & G & BE & 27.4.c & 2020 &   3 & Monitoring started on 3 March and stopped on 19 March. Since 19 March monitoring was not allowed any more due to Covid 19. \\ 
  YserG & G & BE & 27.4.c & 2020 &   3 & Monitoring started on 3 February and stopped on 5 March. On 6 March there was a malfunction at the sluice, after that water level was too high to perform the monitoring and on 19 March monitoring was not allowed any more due to Covid 19. \\ 
  BroG & G & GB & 27.4.c & 2021 &   4 & Provisional data up to July 2021. Trap flooded out May and June.  \\ 
  GiScG & G & FR & 27.8.b & 2021 &   4 & Provisional data \\ 
  SeEAG & G & GB & 27.7.f & 2021 &   3 & This is a provisional figure, with approx. 60\% of returns processed to date, Because of Brexit we shouldn't be using the 2021 series at all. \\ 
  SeHMG & G & GB & 27.7.f & 2021 &   4 & 0.52 was restocked to the �ghome�h rivers; remaining 0.06t exported to Northern Ireland within UK also for restocking \\ 
  VacG & G & FR & 37.1.2 & 2021 &   4 & Provisional data \\ 
   \hline
\end{tabular}
\end{table}

 %====================================
 
 
 
\subsection{Number of series available}


Glass eel and glass eel + young yellow eel time-series available maximum :
41 in
2015
34 in 2020. The maximum number of older
yellow eel time-series has increased to 16 in
2019 but dropped to
7 (Figure \ref{figure2}).

 %========================================== 
\begin{figure}[H]
\centering
\includegraphics[width=0.5\textwidth]{figure2.png}
\caption{Trends in number of glass eel (black circle), glass+young yellow eel
(grey triangle) and older yellow eel (black triangle) time-series giving a
report in  any specific year.}
\label{figure2}
\end{figure}
 %========================================== 
 
 
 
%%%%%%%%%%%%%%%%%%%%%%%%%%%%%%%%%%%%%%%%%%%%%%%%%%%%%%%%%%%%%%%%%%%%%%%%%%%%%%%%%%%%%%
%************************************************************************************
%%%%%%%%%%%%%%%%%%%%%%%%%%%%%%%%%%%%%%%%%%%%%%%%%%%%%%%%%%%%%%%%%%%%%%%%%%%%%%%%%%%%%%
%%%%%%%%%%%%%%%%%%%%%%%%%%%%%%%%%%%%%%%%%%%%%%%%%%%%%%%%%%%%%%%%%%%%%%%%%%%%%%%%%%%%%%
%************************************************************************************
%%%%%%%%%%%%%%%%%%%%%%%%%%%%%%%%%%%%%%%%%%%%%%%%%%%%%%%%%%%%%%%%%%%%%%%%%%%%%%%%%%%%%%



%%%%%%%%%%%%%%%%%%%%%%%%%%%%%%%%%%%%%%%%%%%%%%%%%%%%%%%%%%%%%%%%%%%%%%%%%%%%%%%%%%%%%%
%************************************************************************************
%%%%%%%%%%%%%%%%%%%%%%%%%%%%%%%%%%%%%%%%%%%%%%%%%%%%%%%%%%%%%%%%%%%%%%%%%%%%%%%%%%%%%%



%%%%%%%%%%%%%%%%%%%%%%%%%%%%%%%%%%%%%%%%%%%%%%%%%%%%%%%%%%%%%%%%%%%%%%%%%%%%%%%%%%%%%%
%************************************************************************************
%%%%%%%%%%%%%%%%%%%%%%%%%%%%%%%%%%%%%%%%%%%%%%%%%%%%%%%%%%%%%%%%%%%%%%%%%%%%%%%%%%%%%%

\subsection{Check on series updates for the 2019 analyses[remove this paragraph]}

$41$ time-series  updated to 2020 
$25$ for glass eel \\ 
$9$ forglass + yellow eel  \\
and $7$ for yellow eel \\
$12$ time-series (0 for glass eel,
$3$ for glass + yellow eel and
$9$ for yellow eel) were updated to 2019
only (Table \ref{table_seriesCYm1})).  

$15$ time-series have been stopped or not updated
beyond 2016 ($13$ for glass eel,
$2$ for glass eel + yellow eel and
0 for yellow eel (Table
\ref{table_serieslost})) but are still included in the analysis.
Some have stopped reporting either because of a lack of recruits in the case of
the fishery-based surveys (Ems in Germany, stopped in 2001; Vidaa in Denmark, stopped in 1990),
 a lack of financial support (the Tiber in Italy, 2006) or the introduction of
quota from from 2008 to 2011 that has disrupted the five fishery-based French
time-series . The two English series (FlaE and BeeG) are still operating but
data have not been updated since 2016.


 %====================================
	% latex table generated in R 4.1.2 by xtable 1.8-4 package
% Wed Sep 14 15:31:19 2022
\begin{table}[htbp]
\centering
\caption{Series updated to 2022. Codes for stages are G = glass eel, GY = glass eel + yellow eel, Y = yellow eel, Area NS = North Sea, EE = Elsewhere Europe, Division = FAO marine division. Series ordered by stage and from North to South} 
\label{table_seriesCY}
\begin{tabularx}{\textwidth}{p{1.3cm}p{6.5cm}p{1cm}p{1cm}p{1cm}p{1cm}p{1.4cm}}
  \hline
\scshape{Site} & \scshape{Name} & \scshape{Coun.} & \scshape{Stage} & \scshape{Area} & \scshape{Division} & \scshape{Kept} \\ 
  \hline
RingG & Ringhals scientific survey & SE & G & NS & 27.3.a &   1 \\ 
  YFS2G & IYFS2 scientific estimate & SE & G & NS & 27.3.a &   1 \\ 
  KlitG & Klitmoeller A & DK & G & NS & 27.3.a &   1 \\ 
  SleG & Slette A & DK & G & NS & 27.4.b &   1 \\ 
  NorsG & Nors A & DK & G & NS & 27.3.a &   1 \\ 
  RhIjG & Rhine Ijmuiden scientific estimate & NL & G & NS & 27.4.c &   1 \\ 
  KatwG & Katwijk scientific estimate & NL & G & NS & 27.4.c &   1 \\ 
  StelG & Stellendam scientific estimate & NL & G & NS & 27.4.c &   1 \\ 
  LauwG & Lauwersoog scientific estimate & NL & G & NS & 27.4.b &   1 \\ 
  RhDOG & Rhine DenOever scientific estimate & NL & G & NS & 27.4.c &   1 \\ 
  YserG & IJzer Nieuwpoort scientific estimate & BE & G & NS & 27.4.c &   1 \\ 
  MaigG & River Maigue & IE & G & EE & 27.7.b &   1 \\ 
  BeeG & Beeleigh\_Glass\_$<$80mm & GB & G & NS & 27.4.c &   1 \\ 
  BroG & Brownshill\_Glass\_$<$80mm & GB & G & NS & 27.4.c &   1 \\ 
  FlaG & Flatford\_GE\_$<$80mm & GB & G & NS & 27.4.c &   1 \\ 
  SeEAG & Severn EA commercial catch & GB & G & EE & 27.7.f &   1 \\ 
  VacG & Vaccares & FR & G & EE & 37.1.2 &   1 \\ 
  GiScG & Gironde scientific estimate & FR & G & EE & 27.8.b &   1 \\ 
  OriaG & Oria scientific monitoring & ES & G & EE & 27.8.b &   1 \\ 
  MiSpG & Minho spanish part commercial catch & ES & G & EE & 27.9.a &   1 \\ 
  NaloG & Nalon Estuary commercial catch & ES & G & EE & 27.8.c &   1 \\ 
  AlbuG & Albufera de Valencia commercial catch & ES & G & EE & 37.1.1 &   1 \\ 
  AlCPG & Albufera de Valencia commercial CPUE & ES & G & EE & 37.1.1 &   1 \\ 
  EbroG & Ebro delta lagoons & ES & G & EE & 37.1.1 &   1 \\ 
  MiPoG & Minho portuguese part commercial catch & PT & G & EE & 27.9.a &   1 \\ 
  VerlGY & Verlath Pumping Station & DE & GY & NS & 27.4.b &   1 \\ 
  BrokGY & Broklandsau Pumping Station & DE & GY & NS & 27.4.b &   1 \\ 
  InagGY & River Inagh & IE & GY & EE & 27.7.b &   1 \\ 
  FealGY & River Feale & IE & GY & EE & 27.7.j &   1 \\ 
  LiffGY & Liffey & IE & GY & EE & 27.7.a &   1 \\ 
  BurrGY & Burrishoole & IE & GY & EE & 27.7.b &   1 \\ 
  ShaAGY & Shannon Ardnacrusha trapping all & IE & GY & EE & 27.7.b &   1 \\ 
  CorG & Corrib Galway Weir & IE & GY & EE & 27.7.b &   1 \\ 
  InagG & River Inagh & IE & GY & EE & 27.7.b &   1 \\ 
  ErneGY & Erne Ballyshannon trapping all & IE & GY & EE & 27.7.b &   1 \\ 
  StraGY & Strangford & GB & GY & EE & 27.7.a &   1 \\ 
  BeeGY & Beeleigh\_Elver\_81-120mm & GB & GY & NS & 27.4.c &   1 \\ 
  NmiGY & New Mills Elvers/Yellow $>$80mm & GB & GY & NS & 27.4.c &   1 \\ 
  GreyGY & Greylake\_Elvers/Yellow (mainly yellow$>$120mm with 20-25\% elvers $<$120mm) & GB & GY & EE & 27.7.g &   1 \\ 
  BannGY & Bann Coleraine trapping partial & GB & GY & EE & 27.6.a &   1 \\ 
  BresGY & Bresle & FR & GY & EE & 27.7.d &   1 \\ 
  RonnY & Ronne A  trapping all & SE & Y & NS & 27.3.a &   1 \\ 
  MorrY & Morrumsan  trapping all & SE & Y & NS & 27.3.d &   1 \\ 
  GotaY & Gota Alv  trapping all & SE & Y & NS & 27.3.a &   1 \\ 
  ShaPY & Shannon Parteen trapping partial & IE & Y & EE & 27.7.b &   1 \\ 
  BeeY & Beeleigh\_Yellow\_121mm+ & GB & Y & NS & 27.4.c &   1 \\ 
  BroY & Brownshill\_Yellow\_$>$120mm & GB & Y & NS & 27.4.c &   1 \\ 
  MertY & Thames - Wandle - Merton Abbey Mills & GB & Y & NS & 27.4.c &   1 \\ 
  MillY & Thames - Hogsmill  Middle Mill & GB & Y & NS & 27.4.c &   1 \\ 
  MolY & Thames-Molesey weir & GB & Y & NS & 27.4.c &   1 \\ 
  RodY & Thames - Roding & GB & Y & NS & 27.4.c &   1 \\ 
  FlaY & Flatford Yellow eel $>$120mm & GB & Y & NS & 27.4.c &   1 \\ 
   \hline
\end{tabularx}
\end{table}

 %====================================

 %====================================
	% latex table generated in R 4.1.0 by xtable 1.8-4 package
% Tue Sep 28 08:13:04 2021
\begin{table}[htbp]
\centering
\caption{Series updated to 2020 see table \ref{table_seriesCY)} for codes. Series ordered from North to South} 
\label{table_seriesCYm1}
\begin{tabularx}{\textwidth}{p{1.3cm}p{6.5cm}p{1cm}p{1cm}p{1cm}p{1.4cm}}
  \hline
\scshape{Site} & \scshape{Name} & \scshape{Coun.} & \scshape{Stage} & \scshape{Area} & \scshape{Division} \\ 
  \hline
FlaG & Flatford\_GE\_$<$80mm & GB & G & EE & . \\ 
  BeeG & Beeleigh\_Glass\_$<$80mm & GB & G & EE & . \\ 
  ViskGY & Viskan trapping all & SE & GY & NS & 27.3.a \\ 
  WiFG & Frische Grube & DE & GY & NS & 27.3.b, c \\ 
  WisWGY & Wallensteingraben & DE & GY & NS & 27.3.b, c \\ 
  HellGY & Hellebaekken & DK & GY & NS & 27.3.a \\ 
  GreyGY & Greylake\_Elvers/Yellow (mainly yellow$>$120mm with 10-15\% elvers $<$120mm) & GB & GY & EE & . \\ 
  KavlY & Kavlingean  trapping all & SE & Y & NS & 27.3.b, c \\ 
  LagaY & Lagan  trapping all & SE & Y & NS & 27.3.a \\ 
  DalaY & Dalalven  trapping all & SE & Y & NS & 27.3.d \\ 
  MotaY & Motala Strom  trapping all & SE & Y & NS & 27.3.d \\ 
  GotaY & Gota Alv  trapping all & SE & Y & NS & 27.3.a \\ 
  DoElY & Dove Elde eel ladder & DE & Y & NS & 27.4.b \\ 
  HartY & Harte  trapping all & DK & Y & NS & 27.3.b, c \\ 
  GudeY & Guden AA�c Tange trapping all & DK & Y & NS & 27.3.a \\ 
  RodY & Thames - Roding & GB & Y & EE & . \\ 
  FlaY & Flatford Yellow eel $>$120mm & GB & Y & EE & . \\ 
  FreY & Fremur & FR & Y & EE & . \\ 
   \hline
\end{tabularx}
\end{table}

 %====================================

 %====================================
	% latex table generated in R 4.1.0 by xtable 1.8-4 package
% Thu Sep 30 10:44:54 2021
\begin{table}[htbp]
\centering
\caption{Series stopped or not updated to 2020 see table \ref{table_seriesCY)} for codes. Series ordered by last year} 
\label{table_serieslost}
\begin{tabularx}{\textwidth}{p{1.3cm}p{6.5cm}p{1cm}p{1cm}p{1cm}p{1.4cm}p{1.2cm}}
  \hline
\scshape{Site} & \scshape{Name} & \scshape{Coun.} & \scshape{Stage} & \scshape{Area} & \scshape{Division} & \scshape{Last Year} \\ 
  \hline
YFS1G & IYFS scientific estimate & SE & G & NS & 27.3.a & 1989 \\ 
  VidaG & Vidaa Hoejer sluice commercial catch & DK & G & NS & 27.4.b & 1990 \\ 
  EmsG & Ems Herbrum commercial catch & DE & G & NS & 27.4.b & 2001 \\ 
  TibeG & Tiber Fiumara Grande commercial catch & IT & G & EE & 37.1.3 & 2006 \\ 
  GuadG & Guadalquivir scientific monitoring & ES & G & EE & 27.9.a & 2007 \\ 
  AdCPG & Adour Estuary (CPUE) commercial CPUE & FR & G & EE & . & 2008 \\ 
  AdTCG & Adour Estuary (catch) commercial catch & FR & G & EE & . & 2008 \\ 
  GiCPG & Gironde Estuary (CPUE) commercial CPUE & FR & G & EE & . & 2008 \\ 
  GiTCG & Gironde Estuary (catch) commercial catch & FR & G & EE & . & 2008 \\ 
  LoiG & Loire Estuary commercial catch & FR & G & EE & . & 2008 \\ 
  SevNG & Sevres Niortaise Estuary commercial CPUE & FR & G & EE & . & 2008 \\ 
  VilG & Vilaine Arzal trapping all & FR & G & EE & . & 2015 \\ 
  MorrY & Morrumsan  trapping all & SE & Y & NS & 27.3.d & 2019 \\ 
  RonnY & Ronne A  trapping all & SE & Y & NS & 27.3.a & 2019 \\ 
   \hline
\end{tabularx}
\end{table}

 %====================================


\setlength{\tabcolsep}{4pt} % defaut 6pt (table column separator)

\subsection{Raw data}
 
Calculation of the geometric mean of all time-series \footnote{This figure is given as it
consistent with the trend provided by WGEEL from 2002 to 2006. The scaling is
performed on the 1979-1994 average of each time-series, and 24 time-series without data during that
period are excluded from the analysis. The time-series left out are :
BeeG, BresGY, BroG, BroY, DoElY, FlaG, FreY, GirnY, GreyGY, GuadG, HellGY, InagGY, KlitG, MaigG, MolY, NmiGY, NorsG, OriaG, RodY, SleG, VacG, VerlGY, WiFG, WisWGY}. \\
 is given in (Figures
\ref{figure3} and \ref{figure4}). 

%==========================================
\begin{figure}[htbp]
\centering
\includegraphics[width=0.8\textwidth]{figure3.png}
\caption{Time-series of glass eel and yellow eel recruitment in European rivers
with time-series having data for the 1979-1994 period
(44 sites).
Each time-series has been scaled to its 1979-1994 average. Note the logarithmic scale
on the y-axis. The mean values and their bootstrap confidence interval (95\%) are represented as
black dots and bars. Geometric means are presented in red.}
\label{figure3}
\end{figure}

%==========================================
\begin{figure}[htbp]
\centering
\includegraphics[width=0.8\textwidth]{figure4.png}
\caption{Time-series of glass eel and yellow eel recruitment in Europe
with 44 time-series out of the 95
available to the working group. Each time-series has been scaled to its 1979-1994 average. 
The mean values of combined yellow and glass eel time-series and their bootstrap 
confidence interval (95\%) are represented as black dots and bars.
 The brown line represents the mean value for yellow eel, the blue line represents the mean value for glass eel time-series.
 The range of  these  time-series  is  indicated  by  a  grey  shade.  Note that
 individual time-series  from  Figure \ref{figure3}  were  removed to make the
 mean value more clear.  Note also the logarithmic scale on the y-axis.}
\label{figure4}
\end{figure}
%==========================================


\subsection{GLM based trend}

The WGEEL recruitment index used in the ICES Annual Stock Advice is fitted using
a GLM (Generalised Linear Model) with gamma distribution and a log link:
$glass~eel \sim year:area + site $, where $glass~eel$ is individual glass eel
time-series, including both pure G series and those identified as a mixture of
glass and yel-low eel (G+Y), $site$ is the site monitored for recruitment,
area is either the continental North Sea or Elsewhere Europe, and year is the
year coded as a categorical value.
For yellow eel time-series, only one estimate is provided:
$yellow~eel \sim year + site $. 

The trend is hindcast using the predictions from 1960 onwards for
52 glass eel time-series and from 1950 onwards for
16 yellow eel time-series.
Some zero values have been excluded from the GLM analysis: 
17 for the glass eel model and
21 for the yellow eel model. This
treatment is parcimoniuous and tests shows it has no effect on the trend (ICES,
2017).\\

The reconstructed values are then aggregated using geometric means of the two
reference area (Elsewhere Europe EE, and North Sea NS). The predictions are given 
in reference to the geometric mean of the 1960-1979 period.
Note that the shift from arithmetic to geometric means was done this year
because \textit{post-hoc} model checking confirmed that log-normal (or Gamma
Distribution) and geometric means are the prefered choice. \\

As some of the values were not complete the 2018, the
level of European eel recruitment compared to the 1960-1979 average has changed 
when compared to last year report. The value has decreased from 9.6\% in last
year report to $5.6$\% for the Elsewhere Europe area.  It remains unchanged 
 for the North Sea at $1.4$\%.
 For 2019, data are \textbf{only provisional} and give estimates at
 $0.5$\% for the North Sea and $6.5$\% for the
Elsewhere Europe area, but some of the series are not yet complete (Figure
\ref{figure5} \ref{figure5_with_logscale} (log.scale), Tables \ref{table_glm_glass_eel}).

%==========================================
\begin{figure}[H]
\centering
\includegraphics[width=0.8\textwidth]{figure5_without_logscale_ribbon.png}
\caption{WGEEL recruitment index: estimated (GLM) glass eel
recruitment for the continental North Sea and Elsewhere Europe series with 95 \%
confidence intervals updated to 2020.
  The  GLM ($glass eel \sim area:year+site$)  was  fitted  on
  52 time-series comprising either pure glass eel or a mixture of 
  glass eels and yellow eels. The predictions $p$ have been scaled to the 1960-1979
  average $\bar{p}_{1960-1979}$. In the Baltic area, recruitment occurs in the yellow eel stage only.}
\label{figure5}
\end{figure}
%==========================================

For yellow eel series, the autumn ascent has not been recorded yet and most of
the series have reported data till the middle of the summer. The 2019 yellow eel
index is at $17$\% of
the 1960-1979 baseline. 


%==========================================
\begin{figure}[H]
\centering
\includegraphics[width=0.8\textwidth]{figure6_without_log_scale.png}
\caption {Geometric mean of estimated (GLM) yellow eel recruitment for Europe updated to 2020. The GLM ($yellow eel \sim year+site$) was fitted to
16 yellow eel time-series $p$ and scaled to the
1960-1979 average $\bar{p}_{1960-1979}$.}
\label{figure6}
\end{figure}
%==========================================

\subsection{Is there a positive trend in recruitment ?}
After high levels in the late 1970s, the recruitment declined and has been very
low for all years after 2000. From ICES (2014) onward a change in the
recruitment has been detected. One of the test used to show that change in 2014
was based on SGIPEE(2011) working group. We have used a test only slighly
different from the test used in ICES 2011. The difference is that the model is
now based on individual series as source data, not only the prediction. The model differs from that used by
wgeel also as year is here treated as a continuous value, whereas it is treated
as a factor in the glm for recruitment, and the years are restricted to
decreasing part of the recruitment (after 1980).
\begin{equation}
glass~eel \sim  \alpha_{site} site + \beta_{area} Y_{>=1980} + \gamma_{area} Y_{>2011} + \epsilon,
\end{equation}
where glass eel is the number of glass eel in the glass eel series, either for the Elsewhere Europe or the North Sea time series, $year_{year>1980}$
is a continuous value corresponding to year after 1980, $year_{year>2011}$ is also
a continuous value, $\epsilon$ is a random error with mean 0 and standard deviation sigma, 
and $\alpha_{site}$, $\beta_{area}$ and $\gamma_{area}$ are the estimated
parameters. The parameters $\gamma_{area}$ are higly significant both in the
Elsewhere Europe area (p=$2e-12$
and for the North Sea p=$1e-21$ North Sea.
This result confirms that there has been a change in the recruitment slope.

\textcolor{purple}{Add a table with past coeff, current coeff, and probabilities and diminish the text}

To test whether there is an increase in recruitment since 2011 the slope of
$\beta_{area}+\gamma_{area}$, i.e. the slope of the recent increase in
recruitment is positive, the NULL hypothesis $H0:b>0$ is tested.

To summarize, after high levels in the late 1970s, the recruitment declined and has been very
low for all years after 2000. There has been a change in the trend in 2011, the
recruitment has stopped to decrease, and has been increasing in the period
2011-2019 with a rate significantly different from zero. However,  this increase needs
to be taken very carefully. Firstly , not all series have been reported for 2019
and the results might change  when missing data is incorporated to the analysis. 
Secondly, even if the results would not change, recruitment remains very low
$0.5$\% for the North Sea 
and $6.5$\% for the
Elsewhere Europe area. Recruitment has been continually decreasing
from 1980 to 2011 (31 years), thus, the positive trend should be maintained during more years in order
to consider that the stock is safe. In that sense and finally,  during the 2011-2019 period, 
the maximum values were reached in 2014; but recruitment has been decreasing
since then.


\tiny
%latex.default(gg, rowlabel = "", rowlabel.just = "c", where = "hptb",     cgroup = cgroupdecade, n.cgroup = rep(ncol(gg0), length(cgroupdecade)),     collabel.just = strsplit("c c c c c c c c c c c c c c c",         " ")[[1]], col.just = strsplit("c c c c c c c c c c c c c c c",         " ")[[1]], label = "table_glm_glass_eel", caption = str_c("GLM $glass~eel \\sim year:area + site $ geometric means of predicted values for ",         vv$nb_series_glass_eel, " glass eel series, values given in percentage of the 1960-1979 period."),     file = str_c(tabwd, "/table_glm_glass_eel.tex"))%
\begin{table}[hptb]
\caption{GLM $glass~eel \sim year:area + site $ geometric means of predicted values for 56 glass eel series, values given in percentage of the 1960-1979 period.\label{table_glm_glass_eel}} 
\begin{center}
\begin{tabular}{ccccccccccccccccccccc}
\hline\hline
\multicolumn{1}{c}{\bfseries }&\multicolumn{2}{c}{\bfseries  1960}&\multicolumn{1}{c}{\bfseries }&\multicolumn{2}{c}{\bfseries  1970}&\multicolumn{1}{c}{\bfseries }&\multicolumn{2}{c}{\bfseries  1980}&\multicolumn{1}{c}{\bfseries }&\multicolumn{2}{c}{\bfseries  1990}&\multicolumn{1}{c}{\bfseries }&\multicolumn{2}{c}{\bfseries  2000}&\multicolumn{1}{c}{\bfseries }&\multicolumn{2}{c}{\bfseries  2010}&\multicolumn{1}{c}{\bfseries }&\multicolumn{2}{c}{\bfseries  2020}\tabularnewline
\cline{2-3} \cline{5-6} \cline{8-9} \cline{11-12} \cline{14-15} \cline{17-18} \cline{20-21}
\multicolumn{1}{c}{}&\multicolumn{1}{c}{EE}&\multicolumn{1}{c}{NS}&\multicolumn{1}{c}{}&\multicolumn{1}{c}{EE}&\multicolumn{1}{c}{NS}&\multicolumn{1}{c}{}&\multicolumn{1}{c}{EE}&\multicolumn{1}{c}{NS}&\multicolumn{1}{c}{}&\multicolumn{1}{c}{EE}&\multicolumn{1}{c}{NS}&\multicolumn{1}{c}{}&\multicolumn{1}{c}{EE}&\multicolumn{1}{c}{NS}&\multicolumn{1}{c}{}&\multicolumn{1}{c}{EE}&\multicolumn{1}{c}{NS}&\multicolumn{1}{c}{}&\multicolumn{1}{c}{EE}&\multicolumn{1}{c}{NS}\tabularnewline
\hline
0&$152$&$208$&&$101$&$~99$&&$115$&$81$&&$35$&$15$&&$19.4$&$4.7$&&$~4.8$&$0.7$&&$7.1$&$0.9$\tabularnewline
1&$130$&$117$&&$~55$&$~85$&&$~89$&$58$&&$17$&$~3$&&$~8.8$&$1.0$&&$~3.7$&$0.5$&&$5.4$&$0.6$\tabularnewline
2&$151$&$180$&&$~50$&$109$&&$~92$&$29$&&$22$&$~8$&&$12.3$&$2.6$&&$~4.9$&$0.5$&&$$&$$\tabularnewline
3&$195$&$225$&&$~56$&$~47$&&$~49$&$23$&&$24$&$~7$&&$13.1$&$1.9$&&$~7.1$&$1.7$&&$$&$$\tabularnewline
4&$120$&$116$&&$~83$&$131$&&$~54$&$10$&&$24$&$~7$&&$~7.3$&$0.6$&&$12.0$&$2.5$&&$$&$$\tabularnewline
5&$135$&$~79$&&$~71$&$~54$&&$~52$&$~8$&&$31$&$~5$&&$~7.4$&$1.1$&&$~7.6$&$0.9$&&$$&$$\tabularnewline
6&$~76$&$~88$&&$116$&$~98$&&$~34$&$~8$&&$25$&$~5$&&$~6.0$&$0.5$&&$11.5$&$1.7$&&$$&$$\tabularnewline
7&$~81$&$~98$&&$115$&$~74$&&$~59$&$~9$&&$41$&$~4$&&$~6.4$&$1.3$&&$10.6$&$1.1$&&$$&$$\tabularnewline
8&$129$&$124$&&$110$&$~55$&&$~70$&$~9$&&$16$&$~3$&&$~5.7$&$1.2$&&$10.4$&$1.8$&&$$&$$\tabularnewline
9&$~67$&$~90$&&$147$&$~95$&&$~45$&$~4$&&$19$&$~7$&&$~4.3$&$0.8$&&$~6.2$&$1.4$&&$$&$$\tabularnewline
\hline
\end{tabular}\end{center}
\end{table}

%latex.default(yy, rowlabel = "", rowlabel.just = "c", where = "hptb",     col.just = strsplit("c c c c c c c c c c c c c c c", " ")[[1]],     landscape = FALSE, label = "table_glm_yellow", caption = str_c("GLM $yellow~eel \\sim year + site $ geometric means of predicted values for ",         vv$nb_series_older, " yellow eel series, values given in percentage of the 1960-1979 period."),     file = str_c(tabwd, "/table_glm_yellow.tex"))%
\begin{table}[hptb]
\caption{GLM $yellow~eel \sim year + site $ geometric means of predicted values for 17 yellow eel series, values given in percentage of the 1960-1979 period.\label{table_glm_yellow}} 
\begin{center}
\begin{tabular}{ccccccccc}
\hline\hline
\multicolumn{1}{c}{}&\multicolumn{1}{c}{ 1950}&\multicolumn{1}{c}{ 1960}&\multicolumn{1}{c}{ 1970}&\multicolumn{1}{c}{ 1980}&\multicolumn{1}{c}{ 1990}&\multicolumn{1}{c}{ 2000}&\multicolumn{1}{c}{ 2010}&\multicolumn{1}{c}{ 2020}\tabularnewline
\hline
0&$182$&$166$&$~59$&$99$&$32$&$18$&$12$&$14$\tabularnewline
1&$262$&$181$&$~62$&$41$&$37$&$18$&$30$&$~1$\tabularnewline
2&$252$&$178$&$108$&$52$&$18$&$37$&$15$&$$\tabularnewline
3&$400$&$150$&$135$&$47$&$14$&$24$&$16$&$$\tabularnewline
4&$196$&$~61$&$~65$&$35$&$56$&$25$&$30$&$$\tabularnewline
5&$304$&$115$&$122$&$65$&$13$&$13$&$12$&$$\tabularnewline
6&$135$&$156$&$~38$&$49$&$10$&$17$&$15$&$$\tabularnewline
7&$157$&$111$&$~78$&$47$&$21$&$20$&$12$&$$\tabularnewline
8&$153$&$174$&$~70$&$61$&$17$&$14$&$19$&$$\tabularnewline
9&$334$&$116$&$~58$&$36$&$21$&$~8$&$15$&$$\tabularnewline
\hline
\end{tabular}\end{center}
\end{table}

\normalsize

\clearpage

\section*{Annex}

\subsection{Additional figures}

We provide the same figures as in the main text but Figure \ref{figure3withoutlogscale} 
without log scale for Figure \ref{figure3}. For the prediction, figures with log
scales are provided (Figures \ref{figure5_with_logscale} and
\ref{figure6_with_logscale}).

%==========================================
\begin{figure}[H]
\centering
\includegraphics[width=0.7\textwidth]{figure3withoutlogscale.png}
\caption{Same as figure \ref{figure3} but without log scale}
\label{figure3withoutlogscale}
\end{figure}
%==========================================
%==========================================
\begin{figure}[H]
\centering
\includegraphics[width=0.7\textwidth]{figure5_ribbon.png}
\caption{Same as figure \ref{figure5} but with a log scale.}
\label{figure5_with_logscale}
\end{figure}
%==========================================
%==========================================
\begin{figure}[H]
\centering
\includegraphics[width=0.7\textwidth]{figure6.png}
\caption{Same graph as figure \ref{figure6} but with a log scale.}
\label{figure6_with_logscale}
\end{figure}
%==================================================





 

 \subsection{Notes to ACOM}
 
 \textcolor{Plum}{Were there comments raised last year ?}

 \subsection{Shiny tab on data visualisation}
 
 \dots{}
\end{document}
